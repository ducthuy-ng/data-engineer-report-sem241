Based on the organization’s operational needs and strategic objectives, we identified several key business requirements:
\begin{itemize}
    \item 

Data-driven decision-making: The organization needs to be able to analyze large datasets quickly and efficiently to make informed decisions that improve customer satisfaction, increase revenue, and enhance operational efficiency.
\item Real-time insights: Decision-makers require up-to-date information on sales performance, inventory levels, and customer sentiment to act quickly and adjust strategies when needed.
\item Data visualization: Stakeholders at various levels of the organization need intuitive and actionable insights, which can be best achieved through clear data visualizations and dashboards.
\item Scalability: As the organization expands, its data storage and processing requirements will grow, requiring a technology solution that can scale seamlessly to handle increasing data volumes.
\end{itemize}
To address these business requirements, we assessed the following big data characteristics:
\begin{itemize}
    \item 

Volume: The organization generates a massive amount of data daily, from transactional data in sales to customer feedback and website logs. This requires a solution that can store and process large datasets without performance degradation.
\item Velocity: The speed at which data is generated, particularly from sales transactions, inventory updates, and customer interactions, requires real-time processing and immediate insights.
\item Variety: The data the organization handles is diverse—ranging from structured data such as sales figures to unstructured data like customer feedback and product reviews. This requires flexibility in data management and processing.
\item Value: The organization needs to extract meaningful insights from large datasets to drive revenue growth, improve customer retention, and optimize operations. Without value-driven insights, big data would be just raw information.
\item Visualization: As the organization seeks to make data insights more accessible, the ability to present data in easily interpretable formats such as charts, graphs, and dashboards is key to empowering non-technical stakeholders to make informed decisions.
\end{itemize}
\newpage