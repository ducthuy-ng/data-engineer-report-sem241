\subsection{Google BigQuery}
\begin{figure}[htp]
    \centering
    \includegraphics[width=0.5\linewidth]{images/bigquery.png}
    \caption{Google BigQuery}
\end{figure}

Google BigQuery is a fully managed, serverless data warehouse service that operates on the Google
Cloud Platform. It is designed for real-time analytics and the ability to handle large-scale
datasets with high efficiency and speed. BigQuery allows users to run complex SQL queries on vast
amounts of data, offering unparalleled performance without needing to manage infrastructure. Key
features of BigQuery include:
\begin{itemize}
    \item 

Scalability for large datasets: BigQuery can efficiently handle petabytes of data, making it
suitable for enterprises with massive datasets that need to be processed and analyzed in real-time.
\item Seamless integration with Google Cloud ecosystem: BigQuery integrates easily with other Google
Cloud services, such as Google Data Studio, Google Analytics, and Google Machine Learning services,
enabling a streamlined workflow for data analytics.
\item SQL-based querying: Users can leverage SQL to write queries, making it accessible for data
analysts who may not have deep programming expertise.
\item Cost-effective pricing: With BigQuery, organizations only pay for the queries they run and the
storage they use, eliminating the need to invest in hardware or manage large data centers.
\item Real-time data analytics: BigQuery supports real-time data processing, making it suitable for
environments where up-to-the-minute analytics are critical.
\end{itemize}

\subsection{Apache Spark}
\begin{figure}[htp]
    \centering
    \includegraphics[width=0.5\linewidth]{images/training-spark.png}
    \caption{Apache Spark}
\end{figure}
Apache Spark is an open-source distributed computing framework that is widely used for large-scale
data processing. Unlike traditional batch processing, Spark enables both batch and stream processing
and is known for its speed and flexibility in handling large datasets. Key features of Apache Spark
include:
\begin{itemize}
    \item In-memory processing: Spark processes data in-memory, significantly boosting speed compared to
    traditional disk-based data processing systems. This makes it an ideal choice for iterative machine
    learning tasks and large-scale data transformations.
    \item Versatility: Spark supports a variety of programming languages, including Python, Java, Scala,
    and R, and can integrate seamlessly with other big data tools like Hadoop and Kafka.
    \item Distributed computing: Spark is designed to run on clusters of machines, enabling it to scale
    easily to process petabytes of data.
    \item Machine Learning support: Apache Spark provides a built-in machine learning library, MLlib,
    which allows users to easily apply machine learning algorithms on large datasets.
    \item Real-time streaming: Spark can process real-time streaming data with Spark Streaming, making
    it an excellent choice for applications requiring low-latency processing and data analysis.
\end{itemize}
Together, Google BigQuery and Apache Spark offer a comprehensive suite for big data processing.
BigQuery excels in data warehousing and quick SQL queries, while Apache Spark provides the
flexibility and power for complex, real-time data analytics and machine learning tasks. These
technologies are highly complementary and allow organizations to scale their data infrastructure and
analytics capabilities as needed.\\
In e-commerce context, Google BigQuery is chose based on its scalability as the organization have to deal with excessive amount of data, especially from sale everyday. Moreover, integrating with Google Cloud ecosystem and supporting numerous engines makes BigQuery suitable for various coding skills analyst and ability to adopt with new requirements at effective cost pricing. On the other hand, Apache Spark provides speed and flexibility in real-time streaming, which is one of the business requirements to increase sale revenues. Moreover, with machine learning support, Spark can give prediction for business to quickly adjust the strategy.
